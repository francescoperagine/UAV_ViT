\section{Conclusion}
\label{sec:conclusion}

This study introduces a novel approach to crop yield prediction utilizing a deep learning model to forecast crop yield based on a digital surface model (DSM) image of the field. The researchers employed a ViT model, a transformer-based architecture that has demonstrated efficacy in computer vision tasks. Additionally, they investigated the application of various loss functions, including Huber Loss, Pseudo-Huber Loss, and Log-Cosh Loss, with the goal of improving model performance. Insightful visualizations are employed to support these experiments, aiding in the evaluation of the model's behavior and predictive capabilities.
Despite the promising methodology, the results indicate that the model was unable to achieve a satisfactory R2Score. 

This suboptimal performance is likely attributable to the malformed ground truth and to the limited dataset size, which is insufficient to adequately train a model with 86 million parameters. 

Looking forward, several potential avenues for improvement can be pursued, like the use of a larger dataset, the use of a different model architecture.
