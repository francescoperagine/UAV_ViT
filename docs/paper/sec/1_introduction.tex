\section{Introduction}
\label{sec:introduction}

\subsection{Global challenge}

The global population is currently estimated at around 8 billion people and is projected to reach 9.7 billion by 2050 and 10.4 billion by 2100 \cite{fao2023}. This growth will have a significant impact on agricultural production, which amounted to 2.5\% of the world population with 1.2 billion tons and \$180 billion in 2022. To meet the food needs of the population, production will need to increase by 70\%\cite{unpd22}.

\subsection{Precision agriculture}

The growing demand for food is a major challenge for food production and to meet this demand it is necessary to increase agricultural productivity in a sustainable way. Precision agriculture can help to achieve this goal, by using data on crop characteristics and innovative technologies to help farmers make more informed decisions about how to manage their fields. Precision agriculture is based on the ability to accurately detect the needs of plants within a cultivated field, using a set of sensor technologies that observe their status in a non-invasive manner. This can lead to improved crop yields, reduced environmental impact and increased profitability.

\subsection{Proposed solution}

AI can furtherly boost precision agriculture, increase crop yields and reduce the use of resources. In this paper, we propose an approach based on the use of UAVs powered by Machine Learning (ML), through the use of a ViT architecture that has been enhanced by transfer learning and fine-tuning. UAVs are equipped with numerous sensors making them ideal for quickly collecting mutiple data, whilst ViTs models perform highly accurate quasi-real-time predictions. Therefore it is possible to provide farmers with instant information about the state of the crops in a fully automated and cost-effective way.